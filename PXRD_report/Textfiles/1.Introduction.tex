\section{Introduction}
Determining the crystal structure of materials is important to understand how these materials behave, a crucial part for developing e.g. medicine, and electronics. Common ways to analyse crystal structure is diffraction experiments, where X-ray diffraction is especially well suited because of the small wavelength on the order of the lattice parameters. The experiment conducted in this report specifically uses a type of X-ray diffraction called powder X-ray diffraction.

\section{Theory}
\subsection{Crystal Lattice Structure}
Crystal lattice structures are described by a Bravais lattice \cite{hofmann2015} which is an infinite array of points described by translational operations
\begin{equation}
    \mathbf{R} = n_1\mathbf{a}_1 + n_2\mathbf{a}_2 + n_3\mathbf{a}_3,
\end{equation}
where $\mathbf{a}_i$ are the lattice vectors. Performing a Fourier transform of the real space lattice gives the reciprocal lattice in k-space, also a Bravais lattice, which is described by the reciprocal lattice vectors $\mathbf{b}_j$ defined by $\mathbf{a}_i \cdot \mathbf{b_j} = 2\pi \delta_{ij}$ \cite{hofmann2015}. The reciprocal lattice is useful when performing diffraction experiments since the diffraction pattern is in k-space. 

To calculate the distance between planes the formula 
\begin{equation}
\frac{1}{d^2}=\frac{h^2}{a^2}+\frac{k^2}{b^2}+\frac{l^2}{c^2}
\label{eq:seperation}
\end{equation}
is used.
If we use this formula for a simple cubic with $a=b=c=\SI{0.432}{\nano\m}$ for the plane(111) we get the separation to $d=\SI{0.249}{\nano\m}$. If we do the same for the plane (211) we obtain a separation of $d=\SI{0.176}{\nano\m}$ and for the plane (100) we get $d=\SI{0.432}{\nano\m}$. 

\subsection{Diffraction}
Diffraction is when light passes an obstacle, like an aperture, and the light deviates and spreads out \cite{hofmann2015}. In scattering experiments constructive and destructive interference can be observed which forms diffraction patterns that is analysed. Depending on the size of the obstacle, the wavelength of the light need to be chosen accordingly. For example, to resolve atomic structure the wavelength need to be on the same order as the lattice spacing, which is X-rays \cite{hofmann2015}.

To evaluate diffraction patterns Bragg's law is mainly used, with is defined as
\begin{equation}
    n\lambda=2d\sin \theta,
    \label{eq:Bragg}
\end{equation}
where $n$ is an integer, $\lambda$ is the wavelength of the incoming photon, $d$ is the spacing between the planes and $\theta$ is the scattering angle. Using this we see that if the incoming x-ray has a wavelength of \SI{0.07}{\nano\m} we can at minimum spacing which can be observed is \SI{0.35}{\angstrom}. In order to observe smaller spacings, a higher photon energy would have to be used. 

For example if we have a (111) plane with an  incident angle \( \theta = 11.2^\circ \) when X-rays of wavelength of \( 0.154 \, \text{nm} \) then the side of the side of each unit cell can be calculated as follows: From \autoref{eq:Bragg} we obtain a the distance planer separation to $d=\SI{0.397}{\nano\m}$, from \autoref{eq:seperation} we obtain that a separation of $d=\SI{0.397}{\nano\m}$ leads to a lattice constant of $a=\SI{6.36}{\angstrom}$ for a simple cubic. 

Another example for a orthorhombic unit cell with dimensions $a=\SI{5.74}{\angstrom}$, $b=\SI{7.96}{\angstrom}$ and $c=\SI{4.95}{\angstrom}$ one can calculate the incident angles for the (100), (010), and
(111) planes if the  wavelength is \SI{0.083}{\nano\m}. For the (100) plane \autoref{eq:seperation} gives a planar separation of $d_{(100)}=\SI{0.574}{\nano\m}$, (010) gives $d_{(010)}=\SI{0.796}{\nano\m}$ and (111) gives $d_{(111)}=\SI{0.339}{\nano\m}$. Using \autoref{eq:Bragg} the following angles are found: $\theta_{(100)} = \SI{4.15}{\degree}$, $\theta_{(010)} = \SI{2.99}{\degree}$ and $\theta_{(111)} = \SI{7.03}{\degree}$.

\subsection{Structure Factors}
Due to the structure of different types of lattices, the diffraction pattern may change. This is due to structure function which are defined as 
\begin{equation}
    F(hkl) = \sum_{m} f_m \exp\left( 2\pi i (u_m h + v_m k + w_m l) \right). 
    \label{eq:structurefunction}
\end{equation}
where $f_m$ is the atomic form factor and $(u_m,v_m,w_m)$ are the fractional coordinates of atoms in the unit cell. for a bcc we know that we have the fractional coordinates (0,0,0) for each corner and (1/2,1/2,1/2). and for an fcc we have the (0,0,0) for each corner,  (0,1/2,1/2), (1/2,0,1/2) and (1/2,1/2,0). In the equation above h,k,l is the value of the miller index. Putting all of this together we know that we will obtain diffraction peaks only if $F(hkl)\neq0$ \cite{solidstatephysics2025}. 

From this we can evaluate which planes we will see from bcc and fcc. From \autoref{eq:structurefunction} we can see that the points located at (0,0,0) will always yield a non-zero result. However, if we start looking at the bcc crystal we can obtain the following:
\begin{align}
    F(hkl) &= f_m \exp\left[ 2\pi i \left(0{h}+0{k}+0{l}\right) \right]+ f_m \exp\left[ 2\pi i \left(\frac{h}{2} + \frac{k}{2} + \frac{l}{2}\right) \right]=f_m\left[1+ \exp\left( 2\pi i \frac{h+k+l}{2} \right)\right] \\
    &\text{if $F(hkl)\neq0$ } \implies -1\neq \exp\left( 2\pi i \frac{h+k+l}{2} \right) \implies \frac{h+k+l}{2} \neq n+\frac{1}{2} \text{, where n=1,2,3...} \\
    &\implies h+k+l\neq 2n+1.
\end{align}
Above we used Euler's identity which says $\exp(i\pi (n+1/2)+1=0$. We have thus shown that for the bcc structure wont show any planes where $h+k+l$ are odd. For the fcc we get the following instead:
\begin{align}
    F(hkl) &= f_m + f_m \exp\left[ 2\pi i \left(\frac{h}{2} + \frac{k}{2}\right) \right]+f_m \exp\left[ 2\pi i \left(\frac{h}{2} + \frac{l}{2}\right) \right]+f_m \exp\left[ 2\pi i \left(\frac{k}{2} + \frac{l}{2}\right) \right]\\
    &\text{if $F(hkl)\neq0$ } \implies -1\neq \exp\left[ 2\pi i \left(\frac{h}{2} + \frac{k}{2}\right) \right]+ \exp\left[ 2\pi i \left(\frac{h}{2} + \frac{l}{2}\right) \right]+ \exp\left[ 2\pi i \left(\frac{k}{2} + \frac{l}{2}\right) \right]
\end{align}
As shown before, we can see above if one of $h,k,l$ is even and the other are odd (or vice versa) then we get $-1\neq1-1-1$ which is not allowed. However if all $h,k,l$ are even or odd we obtain $-1\neq1+1+1$ which is true. Thus the (h,k,l) is always all odd or all even for the fcc. 

The order of the peaks we will see will be ordered as 
\begin{equation}
    \sqrt{h^2+k^2+l^2},
    \label{eq:ordering}
\end{equation}  
where h, k and l are the miller indices of that specific plane. Thus, for a bcc crystal the peaks will be in the order (110), (200), (211), (220) and so on. For the fcc the peaks will be (111),(200),(220),(311),(222) and so on.  

\subsection{Powder X-ray Diffraction}
Powder X-ray diffraction (PXRD) is a method of diffraction to analyse crystal structure. In PXRD the sample is a fine powder of many small crystallites instead of a single large crystal \cite{hofmann2015}. This means that the crystal planes will be randomly oriented, and all possible orientations will be present. The diffraction pattern will look like a series of rings, instead of sharp points which arises from single crystal diffraction \cite{hofmann2015}. One of the advantages of powder diffraction is that it is quicker than other methods of analysing crystal structures.


Transmission, fluorescence, 

Xray diffraction, oscillating E field. When X ray hit the electrons the electrons start o vibrate in that frequency
- Constructive and destructive interference 

Bragg's law: $n\lambda=PD=2s=2d\sin\theta$

Crystal lattice: fcc, bcc


In powder diffraction we have all possible orientation sof the crystallites and some will be oriented in the right way for diffraction. Give rise to powder diffraction rings. 

Scherrer's formula
\begin{equation}
    t=\frac{k\lambda}{\beta\cos\theta}
\end{equation}

Structure factor: Se photos. For a simple cubic we see all peaks, for bcc and fcc we loose some peaks. 

