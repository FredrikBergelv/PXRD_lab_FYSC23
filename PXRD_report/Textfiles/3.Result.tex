\newpage
\section{Result}

In \autoref{fig:result_figure} one can see the given spectrum together with gaussian fits of all the peaks. The extracted mean peak angle, the standard deviation and the calculated full with maxima (using \autoref{eq:FWHM}) can bee seen in \autoref{tab:sample1} and \autoref{tab:sample2}. 

\begin{figure}[H]
    \centering
    \begin{subfigure}[b]{0.89\textwidth} 
        \includegraphics[width=\textwidth]{Figures/gaussian_sample1.pdf}
        \subcaption{This is the spectrum from sample 1.}
        \label{fig:subfigure1}
    \end{subfigure}
    \begin{subfigure}[b]{0.89\textwidth} 
        \includegraphics[width=\textwidth]{Figures/gaussian_sample2.pdf}
        \subcaption{This is the spectrum from sample 2.}
        \label{fig:subfigure2}
    \end{subfigure}
    \caption{These plots display the spectrum gathered from the detector. It is evident that the sharp peaks in both plots are caused by hot pixels, damaged detector elements that produce spurious signals.}
    \label{fig:result_figure}
\end{figure}

\begin{table}[H]
    \centering
    \caption{Fitting results for Sample 1}
    \begin{tabular}{cccc}
    \toprule
    Peak & Angle $\theta$ (radians) & Standard deviation $\sigma$ (radians) & FWHM (radians) \\
    \midrule
    Peak 1 & \num{0.15362(0.00010)}{} & \num{0.00165(0.00010)} & \num{0.00388(0.00023)}{} \\
    Peak 2 & \num{0.17731(0.00019)}{} & \num{0.00227(0.00020)}{} & \num{0.00535(0.00046)}{} \\
    Peak 3 & \num{0.25180(0.00018)}{} & \num{0.00234(0.00018)}{} & \num{0.00550(0.00043)}{} \\
    Peak 4 & \num{0.29569(0.00015)}{} & \num{0.00225(0.00015)}{} & \num{0.00530(0.00035)}{} \\
    Peak 5 & \num{0.30900(0.00015)}{} & \num{0.00261(0.00021)}{} & \num{0.00616(0.00049)}{} \\
    \bottomrule
    \end{tabular}
    \label{tab:sample1}
\end{table}

\begin{table}[H]
    \centering
    \caption{Fitting results for Sample 2}
    \begin{tabular}{cccc}
    \toprule
    Peak & Angle $\theta$ (radians) & Standard deviation $\sigma$ (radians) & FWHM (radians) \\
    \midrule
    Peak 1 & \num{0.17746(0.00012)}{} & \num{0.00172(0.00014)}{} & \num{0.00404(0.00034)}{} \\
    Peak 2 & \num{0.25186(0.00017)}{} & \num{0.00380(0.00039)}{} & \num{0.00895(0.00092)}{} \\
    Peak 3 & \num{0.30942(0.00014)}{} & \num{0.00326(0.00024)}{} & \num{0.00768(0.00056)}{} \\
    \bottomrule
    \end{tabular}
    \label{tab:sample2}
\end{table}


Using peak 1 from the first sample and $n=1$, the plane (111) we obtain a spacing $a=\SI{4.0213(0.0026)}{\angstrom}$ which coincides with the material Au, with $a=\SI{4.065}{\angstrom}$, using \autoref{eq:Bragg} and \autoref{eq:seperation} with the given table \cite{solidstatephysics2025}. The first peak coincides with the plane (111) as seen in \autoref{eq:ordering} and all indices are odd which is must be true since Au is a FCC. This verifies this result. 

Using peak 2 from the second sample and $n=1$, the plane (110) we obtain a spacing $a=\SI{2.8484(0.0030)}{\angstrom}$ which coincides with the material Fe, with $a=\SI{2.856}{\angstrom}$, using \autoref{eq:Bragg} and \autoref{eq:seperation} with the given table \cite{solidstatephysics2025}. The second peak coincides with the plane (110) as seen in \autoref{eq:ordering}, also the sum of the miller indices is even which must be true for a BCC such as Fe, verifying this result. 


Using \autoref{eq:Scherre}, we obtain the following particle sizes as shown in \autoref{tab:Scherre1} and \autoref{tab:Scherre2}. The mean particle size for Sample 1 is \SI[scientific-notation=false]{13.44(0.4)}{\nano\meter}, and for Sample 2, it is \SI[scientific-notation=false]{11.2(0.6)}{\nano\meter}.


\begin{table}[H]
    \centering
    \begin{minipage}{0.45\textwidth}
        \centering
        \caption{Powder size $t$ for Sample 1}
        \begin{tabular}{cc}
        \toprule
        Peak & Size (\SI{}{\m}) \\
        \midrule
        Peak 1 & \num{17.4(10)e-8} \\
        Peak 2 & \num{12.7(11)e-8} \\
        Peak 3 & \num{12.5(10)e-8} \\
        Peak 4 & \num{13.2(9)e-8} \\
        Peak 5 & \num{11.4(9)e-8} \\
        \bottomrule
        \end{tabular}
        \label{tab:Scherre1}
    \end{minipage}%
    \hfill
    \begin{minipage}{0.45\textwidth}
        \centering
        \caption{Powder size $t$ for Sample 2}
        \begin{tabular}{cc}
        \toprule
        Peak & Size (\SI{}{\m}) \\
        \midrule
        Peak 1 & \num{1.68(14)e-8} \\
        Peak 2 & \num{7.7(8)e-9} \\
        Peak 3 & \num{9.1(7)e-9} \\
        \bottomrule
        \end{tabular}
        \label{tab:Scherre2}
    \end{minipage}
\end{table}