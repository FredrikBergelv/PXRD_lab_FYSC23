\section*{Questions (we can remove this later but keep it now for keeping track)}

\begin{enumerate}
    \item Calculate the separations of the planes (111), (211), and (100) in a crystal in which the simple cubic unit cell has the side \( a = 0.432 \, \text{nm} \). \textbf{DONE}
    \item From Bragg’s equation, what is the minimum d-spacing that can be observed with a wavelength of \( 0.07 \, \text{nm} \)? What do you need in order to observe smaller d spacings? \textbf{DONE}
    \item Reflection from the (111) planes of a cubic crystal was observed at an incident angle \( \theta = 11.2^\circ \) when Cu K\(\alpha\) X-rays of wavelength \( 0.154 \, \text{nm} \) were used. What is the length of the side of the unit cell? \textbf{DONE}
    \item Calcium carbonate crystals in the form of aragonite have an orthorhombic unit cell with dimensions \( a = 0.574 \, \text{nm} \), \( b = 0.796 \, \text{nm} \), \( c = 0.495 \, \text{nm} \). Calculate the incident angles for the (100), (010), and (111) reflections using radiation of wavelength \( 0.083 \, \text{nm} \). \textbf{DONE}
    \item Calculate the allowed values of \( hkl \) for a body-centered cubic (bcc) and face-centered cubic (fcc) system that will give Bragg points on a diffraction pattern. \textbf{DONE}
    \item Find the two elements. \textbf{DONE}
    \item Measure the full width at half maximum (FWHM) of each peak of the two
    diffraction patterns and calculate the crystallite size using Scherrer’s equation. Scherrer’s
    equation is suitable for particle size below approximately 100 nm. Which property of
    light limits this kind of quantification? Hint: constructive interference of two waves (and
    therefore diffraction) does not happen with any kind of light
    \item Bragg’s law is based on the ideal case where a strictly monochromatic source-
    is used. However, every X-ray source has a certain finite spectral bandwidth, $\Delta\lambda/\lambda$. As
    a result, the broader the spectral bandwidth, the broader the diffraction peaks. 
    \begin{itemize}
        \item \textbf{Hint:} Constructive interference of two waves (and therefore diffraction) does not happen with any kind of light.
    \end{itemize}
    Bragg’s law is based on the ideal case where a strictly monochromatic source is used. However, every X-ray source has a certain finite spectral bandwidth, \( \frac{\Delta \lambda}{\lambda} \). As a result, the broader the spectral bandwidth, the broader the diffraction peaks.
    
    \item[(a)] Prove that the spectral contribution of the source to the peak width, \( \Delta \theta \), is related to the spectral bandwidth through the following relation:
    \[
    \Delta \theta = \tan \theta \frac{\Delta \lambda}{\lambda} \quad (9.2)
    \]
    \textbf{Hint:} Differentiate Bragg’s law.

    \item[(b)] Although this contribution affected the estimation you made in Exercise 7, it can be quantified to obtain a more exact crystallite size. Can you think of at least one other phenomenon or instrumental property that can affect the peak widths?
\end{enumerate}

\section{Result}




\begin{figure}[H]
    \centering
    \begin{subfigure}[b]{0.99\textwidth} 
        \includegraphics[width=\textwidth]{Figures/gaussian_sample1.pdf}
        \subcaption{This is the first subfigure.}
        \label{fig:subfigure1}
    \end{subfigure}
    \begin{subfigure}[b]{0.99\textwidth} 
        \includegraphics[width=\textwidth]{Figures/gaussian_sample2.pdf}
        \subcaption{This is the second subfigure.}
        \label{fig:subfigure2}
    \end{subfigure}
    \caption{This figure shows two subfigures with separate captions.}
    \label{fig:result_figure}
\end{figure}
\begin{table}[H]
    \centering
    \caption{Fitting results for Sample 1}
    \begin{tabular}{cccc}
    \toprule
    Peak & Angle $\theta$ (radians) & Standard deviation $\sigma$ (radians) & FWHM (radians) \\
    \midrule
    Peak 1 & \num{0.15362(0.00010)}{} & \num{0.00165(0.00010)} & \num{0.00388(0.00023)}{} \\
    Peak 2 & \num{0.17731(0.00019)}{} & \num{0.00227(0.00020)}{} & \num{0.00535(0.00046)}{} \\
    Peak 3 & \num{0.25180(0.00018)}{} & \num{0.00234(0.00018)}{} & \num{0.00550(0.00043)}{} \\
    Peak 4 & \num{0.29569(0.00015)}{} & \num{0.00225(0.00015)}{} & \num{0.00530(0.00035)}{} \\
    Peak 5 & \num{0.30900(0.00015)}{} & \num{0.00261(0.00021)}{} & \num{0.00616(0.00049)}{} \\
    \bottomrule
    \end{tabular}
\end{table}

\begin{table}[H]
    \centering
    \caption{Fitting results for Sample 2}
    \begin{tabular}{cccc}
    \toprule
    Peak & Angle $\theta$ (radians) & Standard deviation $\sigma$ (radians) & FWHM (radians) \\
    \midrule
    Peak 1 & \num{0.17746(0.00012)}{} & \num{0.00172(0.00014)}{} & \num{0.00404(0.00034)}{} \\
    Peak 2 & \num{0.25186(0.00017)}{} & \num{0.00380(0.00039)}{} & \num{0.00895(0.00092)}{} \\
    Peak 3 & \num{0.30942(0.00014)}{} & \num{0.00326(0.00024)}{} & \num{0.00768(0.00056)}{} \\
    \bottomrule
    \end{tabular}
\end{table}

Using peak 1 from the first sample and $n=1$, the plane (111) we obtain a spacing $a=\SI{4.0213(0.0026)}{\angstrom}$ which coincides with the material Au using \autoref{eq:Bragg} and \autoref{eq:seperation} with the given table \cite{solidstatephysics2025}. The first peak coincides with the plane (111) as seen in \autoref{eq:ordering} and all indices are odd which is must be true since Au is a fcc. This verifies this result. 

Using peak 2 from the second sample and $n=1$, the plane (110) we obtain a spacing $a=\SI{2.8484(0.0030)}{\angstrom}$ which coincides with the material Fe using \autoref{eq:Bragg} and \autoref{eq:seperation} with the given table \cite{solidstatephysics2025}. The second peak coincides with the plane (110) as seen in \autoref{eq:ordering}, also the sum of the miller indices is even which must be tru for a bcc such as Fe, verifying this result. 


Using \autoref{eq:Scherre}, we obtain the following particle sizes as shown in \autoref{tab:Scherre1} and \autoref{tab:Scherre2}. The mean particle size for Sample 1 is \SI[scientific-notation=false]{13.4(0.4)}{\nano\meter}, and for Sample 2, it is \SI[scientific-notation=false]{11.2(0.6)}{\nano\meter}.


\begin{table}[H]
    \centering
    \begin{minipage}{0.45\textwidth}
        \centering
        \caption{Powder size $t$ for Sample 1}
        \begin{tabular}{cc}
        \toprule
        Peak & Size (\SI{}{\m}) \\
        \midrule
        Peak 1 & \num{17.4(10)e-8} \\
        Peak 2 & \num{12.7(11)e-8} \\
        Peak 3 & \num{12.5(10)e-8} \\
        Peak 4 & \num{13.2(9)e-8} \\
        Peak 5 & \num{11.4(9)e-8} \\
        \bottomrule
        \end{tabular}
        \label{tab:Scherre1}
    \end{minipage}%
    \hfill
    \begin{minipage}{0.45\textwidth}
        \centering
        \caption{Powder size $t$ for Sample 2}
        \begin{tabular}{cc}
        \toprule
        Peak & Size (\SI{}{\m}) \\
        \midrule
        Peak 1 & \num{1.68(14)e-8} \\
        Peak 2 & \num{7.7(8)e-9} \\
        Peak 3 & \num{9.1(7)e-9} \\
        \bottomrule
        \end{tabular}
        \label{tab:Scherre2}
    \end{minipage}
\end{table}